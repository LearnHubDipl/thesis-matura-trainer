% !TEX root = thesis.tex
\documentclass[12pt,a4paper,titlepage,listof=totoc,bibliography=totoc,chapteratlists=0pt]{scrreprt}
\newcommand{\thesislang}{de} 
\newcommand{\thesistitle}{HTL Maturatrainer}
\newcommand{\department}{Medientechnik} 

\newcommand{\firstauthor}{Thomas Gossenreiter}
\newcommand{\secondauthor}{Stefanie Steinmair}

\newcommand{\duedateen}{April 6, 2026} 
\newcommand{\duedatede}{6. April 2026} 
\newcommand{\supervisor}{Birgit Schröder}
\input{header}

\makeatletter
\def\bstctlcite{\@ifnextchar[{\@bstctlcite}{\@bstctlcite[@auxout]}}
\def\@bstctlcite[#1]#2{\@bsphack
	\@for\@citeb:=#2\do{%
		\edef\@citeb{\expandafter\@firstofone\@citeb}%
		\if@filesw\immediate\write\csname #1\endcsname{\string\citation{\@citeb}}\fi}%
	\@esphack}
\makeatother

\clubpenalty=10000 
\widowpenalty=10000
\displaywidowpenalty=10000
\interfootnotelinepenalty=10000

\title{\thesistitle}
\makeatletter
\@ifundefined{fourthauthor}{
  \@ifundefined{thirdauthor}{
    \author{\firstauthor, \secondauthor}
  }{
    \author{\firstauthor, \secondauthor, \thirdauthor}
  }
}{
  \author{\firstauthor, \secondauthor, \thirdauthor, \fourthauthor}
}
\makeatother

\makeindex
\makeglossaries
\begin{document}
\bstctlcite{IEEEexample:BSTcontrol}
\newcommand{\reminder}[1]
{ \textcolor{red}{<[{\bf\marginpar{\mbox{$<==$}} #1 }]>} }
\newcommand{\icode}[1]{\lstinline$#1$}
%\urlstyle{same}
%\setstretch{1.5}
\setstretch {1.433}
\renewcommand{\arraystretch}{1.2}

\includepdf{./titlepage/coversheet}
\pagenumbering{Roman}
\newpage
\input{oath}
\input{./sections/abstract}

\pagestyle{plain}

\IfStrEq{\thesislang}{de}
{
	\renewcommand{\lstlistlistingname}{Quellcodeverzeichnis}
}
{
 % keep at list of listing
}

\setcounter{tocdepth}{3}  % Subsubsections im TOC
\tableofcontents
\newpage
\setcounter{RPages}{\value{page}}
\setcounter{page}{0}
\pagenumbering{arabic}
\pagestyle{scrheadings}

\begin{spacing}{1}

% =========================
% Kapitel 1 – Einleitung
% =========================
\chapter{Einleitung}

\section{Ausgangslage}
\textit{Beschreibung der aktuellen Situation: Warum besteht Bedarf für die Matura-Lernplattform? Probleme beim Lernen, fehlendes Feedback, etc.}

\section{Motivation und Problemstellung}
\textit{Warum ist das Thema interessant und relevant? Hauptproblem, das gelöst werden soll.}

\section{Zielsetzung}
\textit{Was soll am Ende entstehen? Funktionen und Ziele der Plattform.}

\section{Grundlagen und Analyse}

\subsection{Marktanalyse}
\textit{Analyse bestehender Lernplattformen und digitaler Angebote.}

\subsubsection{StudySmarter}
\textit{Funktionsweise, Zielgruppe, Vorteile/Schwächen, Vergleich mit eigenem Projekt.}

\subsection{Anforderungen eines digitales Maturatraining}
\textit{Funktionale und nicht-funktionale Anforderungen.}

\subsection{Nutzen für Schüler:innen}
\textit{Mehrwert der Plattform: Motivation, individuelles Lernen, Feedback, Vorbereitung.}

\section{Lernstrategien}
\setauthor{\secondauthor}

\subsection{Leitner-System}
\textit{Karteikarten-Prinzip, Vorteile, Umsetzung im Projekt.}

\subsection{Spaced Repetition}
\textit{Verteiltes Lernen, Effektivität, Umsetzung digital.}

\subsection{Vergleich verschiedener Lernstrategien und Schlussfolgerung}
\textit{Leitner, Spaced Repetition, klassisches Lernen – Vor- und Nachteile - Ergebnis.}

% =========================
% Kapitel 2 – Technologische Entscheidungen
% =========================
\chapter{Technologische Entscheidungen und Architektur}

\section{Backend}
\textit{Backend-Aufbau}

\subsection{Datenbankstruktur}
\textit{Datenbanktabellen}

\subsection{Quarkus}
\textit{REST-Endpunkte, Speicherung von Feedback/Statistiken.}

\section{Frontend – Angular}

\subsection{Aufbau und Komponenten}
\textit{Struktur der App: Login, Dashboard, Übungsmodus, Prüfung, Ergebnisübersicht. Komponenteninteraktion.}

\section{Entwicklungsumgebung und Versionsverwaltung}
\textit{Verwendete Tools (IntelliJ, GitHub, Docker), Branching-Strategie, Teamwork.}

% =========================
% Kapitel 3 – Praktische Umsetzung
% =========================
\chapter{Praktische Umsetzung}

\section{Usecases}

\subsection{Üben}
\textit{User (Schüler) möchte für gezielte Themen für die Matura lernen.}
\textit{Usecase-Diagramme für User (Schüler) → Fragen beantworten, Feedback erhalten, eigenen Fragenpool erstellen}

\subsection{Prüfungssimulationen}
\textit{User (Schüler) möchte eine Prüfungssimulation starten, um sein Wissen zu testen.}
\textit{Usecase-Diagramme für User (Schüler) → Fragen beantworten, Feedback am Ende erhalten, Note erhalten}

\subsection{Fortschrittsübersicht}
\textit{User (Schüler) möchte seinen Lernfortschritt jederzeit einsehen und Statistiken betrachten.}
\textit{Usecase-Diagramme für User (Schüler) → Fortschritt (\% der unbeantworteten, falsch beantworteten, 1x, 2x, genug beantworteten Fragen), Durchschnittswertung bei Prüfungen usw.}

\section{Benutzeroberfläche und Design}

\subsection{Mockups und User Flow}
\textit{UI-Entwürfe (Figma), Nutzerführung Start → Übungsmodus → Ergebnis. Fokus auf Usability.}

% ------------------------------------------
% Use Case: Üben
% ------------------------------------------
\section{Use Case: Üben}

\subsection{Fragenhandling und Fragetypen}
\setauthor{\firstauthor}
\textit{Speicherung, Abruf, Darstellung verschiedener Fragetypen, Feedbacksystem.}

\subsection{Übungsmodus mit direktem Feedback}
\setauthor{\firstauthor}
\textit{Technische Umsetzung, Ablauf, User Experience.}

\subsection{Lösungswege und Nutzerinteraktion}
\setauthor{\secondauthor}
\textit{Einsehen von Lösungswegen, Feedbackmechanismen}

\subsection{Verwaltung von Fragenpools}
\setauthor{\secondauthor}
\textit{Auswahl eigener Fragen, Erstellung von individuellen Pools, Backend- und Frontend-Umsetzung.}

% ------------------------------------------
% Use Case: Prüfungssimulation
% ------------------------------------------
\section{Use Case: Prüfungssimulation}

\subsection{Prüfungsmodus mit Zeitlimit}
\setauthor{\firstauthor}
\textit{Simulation der Prüfung, Zeitlimit, realistische Prüfungssituation.}

\subsection{Bewertung und Ergebnisausgabe}
\setauthor{\firstauthor}
\textit{Ergebnisanzeige am Ende, Notenberechnung, Auswertung.}

% ------------------------------------------
% Use Case: Fortschritt
% ------------------------------------------
\section{Use Case: Fortschritt einsehen \& analysieren}

\subsection{Lernstrategien und Fehleranalyse}
\setauthor{\secondauthor}
\textit{Identifikation von Wiederholungsbedarf und Schwachstellen.}

\subsection{Spaced Repetition nach Leitner}
\setauthor{\secondauthor}
\textit{Technische Umsetzung der Lernstrategie, Speicherintervalle, Fortschrittslevel.}

\subsection{Lernstatistiken und Visualisierung}
\setauthor{\secondauthor}
\textit{Erhebung und grafische Darstellung der Daten zur Nutzerleistung. Bibliotheken und Darstellungskonzepte.}

\section{Usability-Test mit Mitschülern}
\textit{Testbeschreibung, Szenarien, Feedback, Umsetzung von Verbesserungen.}

\end{spacing}



\newpage
\pagenumbering{Roman}
\setcounter{page}{\value{RPages}}
\input{glossary}
\glsnogroupskiptrue
\IfStrEq{\thesislang}{de}
{
	\printglossary[title=Glossar,toctitle=Glossar]
}
{
	\printglossary[title=Glossary,toctitle=Glossary]
}
\spacing{1}{
\IfStrEq{\thesislang}{de}
{
	\bibliographystyle{ieeetrande}
}
{
	\bibliographystyle{IEEEtran}
}
\bibliography{bib}
}
\listoffigures
\listoftables
\lstlistoflistings
\appendix
\addchap{Anhang}
\input{./sections/appendix}
\end{document}

